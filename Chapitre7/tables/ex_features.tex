\begin{table}[]
    \centering
    \begin{tabular}{|l|c|}
      \hline
    \textbf{Caractéristiques} & \textbf{Nombre d'occurrences} \\\hline
    Répétition d'un mot (deg1)     & 26 \\
    Répétitions de deux mots (deg2) & 4\\
    Pauses dans le discours (\textit{euh, bah, hein, eh, etc.}) & 22\\
    Marqueurs forts (\textit{important, inquiet, scandaleux, etc.}) & 14\\
    Marqueurs faibles (\textit{quand même, franchement, etc.}) & 3\\
    Négations (\textit{pas, ne, n'}) & 30\\
    \textit{c'est} & 44\\ \hline
    nombre de mots dans \textit{lettre recommandée} & 1050 \\
    nombre de segment de parole dans \textit{lettre recommandée} & 152 \\ \hline
    \end{tabular}
    \caption{Sept caractéristiques et leur nombre d'occurrences permettent de modéliser les indices supposés être responsables de la frustration dans les conversations. Le nombre total de mots et de segments de parole de la conversation appelée \textit{lettre recommandée}, sont également indiqués.}
    \label{tab:ex_features}
\end{table}
