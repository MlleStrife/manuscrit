\begin{table}[htp!]
   \centering
   \begin{tabular}{|l| l  l  l|}
       \hline
                   &activation    &valence   &liking \\
       \hline
       %44kHz           &.322 (.415)  &.337 (.440) &.106 (.209) \\
       %8kHz            &.486 (.539) &.495 (.532) &.187 (.289) \\
       %8kHz + noise    &.471 (.497) &.464 (.486) &.204 (.250) \\
       44kHz           &\textbf{.528}  &\textbf{.515} &\textbf{.304} \\
       8kHz            &.486 &.495 &.187 \\
       8kHz + bruit    &.471 &.464 &.204 \\
       \hline
   \end{tabular}
   \caption{Comparaison des scores de développement entre les versions originales du corpus SEWA et les versions dégradées. Les dégradations sont le sous-echantillonage des enregistrement audio et l'ajout de différents bruits. Nous utilisons les descripteurs eGeMAPS-47 ainsi qu'un modèle CNN. L'entrainement et les prédictions sont effectuées sur les conversations allemandes.}
   \label{tab:downsample}
\end{table}
