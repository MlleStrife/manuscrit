\chapter*{Introduction}
\addcontentsline{toc}{chapter}{Introduction}
\chaptermark{Introduction}

\subsubsection{Contexte}

Nous apprenons dès le plus jeune age que la communication est l'outil par excellence pour vivre en société. Cette communication, pourtant si étudiée par tout le monde, reste aujourd'hui pleine de mystère. Le nombre de domaine de recherche qui gravite autour de la communication en est le marqueur plus qu'évident. Quand nous pensons à la communication, nous pensons tout d'abord à l'oral, à l'écrit, aux mots et aux phrases. Puis si nous creusons un peu, nous pensons aux gestes, aux visages, aux us et coutumes de chaque société. Mais la communication ne se réduit pas à ces quelques canaux. L'émotion est une part importante de tout être, elle nous définit à un instant t en tant qu'individu. Il est donc capital qu'elle puisse être comprise par autrui, qu'elle puisse être communiquée. Nous sommes tous capable de ressentir des émotions qui impactent nos moyens de communiquer et de ce fait qui peuvent être reconnues par nos interlocuteurs.

La parole est un canal de communication où il est possible d'exprimer de nombreux messages. Il s'agit d'un de nos canaux principaux pour vivre en société, puisque c'est par elle que nous nous exprimons le plus et le plus clairement. La parole véhicule aussi les émotions qui nous traversent. Nous sommes capable de reconnaître la détresse d'un individu à son ton de voix et donc de nous montrer empathique quand l'autre en a besoin. Cette reconnaissance des émotions d'autrui, nous permet alors de mieux adapter notre discours, de mieux réagir et donc d'agir en tant que membre de la société.

Il est donc tout aussi important pour des industriels, de pouvoir reconnaître l'état émotionnel de ces interlocuteurs, afin d'adapter leur discours. En effet, toute entreprise privée a pour objectif de gagner de l'argent. Pour y parvenir, elle doit acquérir et conserver des clients qui vont lui acheter son ou ses produits. Il est donc primordial dans ce contexte d'avoir une connaissance, même sommaire, des émotions de ces clients. Mon sujet de thèse s'inscrivant dans un besoin industriel bien particulier, nous avons réfléchi aux émotions qui avaient une grande importance dans une relation client-entreprise.

Travaillant main dans la main avec des centres d'appels, il nous est paru évident que la plupart des clients n'appellepas pour signifier leur joie, leur peine ou même leur contentement. Non, la plupart d'entre nous, quand nous téléphonons à ces numéros pas forcément gratuits, c'est parce que nous avons besoin d'information ou alors parce que quelque chose ne va pas. Dans ce second cas, il nous semble que la frustration correspond à l'émotion sur laquelle nous devons axer nos recherches. De même, il nous semblait important de ne pas recueillir uniquement que le négatif, nous voulions aussi permettre aux clients contents d'être entendus. C'est pour cela que nous avons également considéré la satisfaction. Ces deux émotions, que nous considérons comme deux facettes d'une même pièce, sont importantes dans la relation clientèle et elles sont étroitement liées à la perte ou au gain d'un client.

C'est dans ce contexte que nous avons voulu inscrire nos travaux : la reconnaissance des états émotionnels, et donc notamment la satisfaction et la frustration, dans des échanges entre des clients et des représentants de diverses entreprises, c'est-à-dire des conseillers de centre d'appels.

Dans ce cas, pourquoi vouloir automatiser la détection de ces émotions, alors que l'humain est tout à fait capable de les détecter ? Aujourd'hui les centres d'appels sont d'immenses plateformes et les conseillers qui travaillent dans ce secteur se comptent en milliers quand aux appels, ils se comptent en centaines par heures. Ces immenses quantités d'appel ne peuvent pas être traitées par l'humain sans devoir employer une quantité pharamineuse de personne. De même, ces conseillers ont beaucoup de travail, il ne serait pas pertinent de rajouter une charge d'annotation de l'émotion en plus. Surtout que ce n'est ni une tâche facile et ni une tâche sans ambiguïté. Même avec une formation adéquate, il est peu probable que tous les annotateurs se rangent à la même annotation.

D'aussi gros volume de données nous font tout de suite penser au machine learning. Ces outils, dont le deep learning fait partie, sont spécialement équipés pour faire face à un tel problème. Depuis maintenant quelques années, nous avons appris à apprendre aux ordinateurs comment faire certaines de nos tâches. Dans le domaine de la parole, on pense notamment à l'ASR (Automatic Speech Recognition), qui utilise des systèmes intelligents pour traduire un signal audio en transcription textuelle. Dans notre étude, nous allons nous inscrire dans le domaine du SER (Speech Emotion Recognition), afin de traduire un signal audio en états émotionnels.

Cette traduction ainsi effectuée, les entreprises peuvent alors détecter les états émotionnels de leur client et ajuster leur discours. En extrapolant, on peut voir des campagnes de réduction envoyées aux clients frustrés, afin de les satisfaire par exemple.

Qu'est ce qui est novateur dans nos travaux ? Tout d'abord nous travaillons sur deux états émotionnels peu dotés, surtout si on considère les données en langue française. Nous démontrons que nous sommes capables de les détecter automatiquement à l'aide de systèmes construits avec des réseaux de neurones. Nous étudions les meilleures représentations de la parole afin d'extraire la frustration et la satisfaction. Enfin, nous tentons d'apporter une justification au succès de ces systèmes.

\subsubsection{Problématiques}
Concrétement, nous avons abordé ces différentes problématiques tout au long de cette thèse :
\begin{itemize}
  \item Quelles sont les émotions que l'on peut retrouver dans les conversations des centres d'appels, qui seront utiles pour les industriels ?
  \item Comment construire des systèmes efficaces de reconnaissance des émotions de la parole ?
  \item Dans quelles modalités de la parole ces émotions peuvent-elles s'exprimer ?
  \item Peut-on utiliser des descripteurs pré-entraînées extraites avec une méthode auto-supervisée pour palier au manque de données ?
  \item Quels sont les marqueurs de satisfaction et de frustration que l'on peut retrouver dans la modalité linguistique ?
  \item Doit on considérer que l'émotion est décrite par la fusion de plusieurs annotations ou peut on prendre en compte les annotations individuelles de chaque personne ?
\end{itemize}

Tous ces questionnements sont traités dans les différents chapitres de contributions de cette thèse.

\subsubsection{Organisation du document}

Ce document est divisé en deux parties. La première partie correspond à une présentation plus détaillée des contextes de cette thèse par un état de l'art des domaines proches de notre travail. La deuxième partie du document recense les contributions que nous avons apportées au sein de cette thèse.

Le premier chapitre traite de l'état de l'art des émotions. Celui-ci est à la croisée de nombreux domaines de recherche, puisque les émotions sont définies par un ensemble de concepts qui continuent d'évoluer. Nous commençons par une définition de l'émotion et par les théories qui ont eu pour but de l'expliquer au fil des époques. Ensuite nous traitons de la représentation de ces émotions que nous utilisons en tant qu'informaticien afin d'en automatiser le traitement. Nous parlons alors des marqueurs des émotions, que ce soit dans la parole, dans le texte ou dans les traits humains. Ce chapitre nous permet de nous placer dans un contexte défini en ce qui concerne les émotions et la définition à laquelle nous nous confortons.

Le deuxième chapitre traite de l'état de l'art de l'apprentissage automatique. Comme nous travaillons sur la parole, nous avons surtout insisté sur les ensembles de méthodologies utilisées pour traiter la parole. Après avoir défini l'apprentissage automatique, nous passons en revue quelques grandes familles d'algorithmes qui permettent cet apprentissage. Nous nous intéressons particulièrement aux réseaux de neurones, à ces spécificités et à quelques architectures établies dans le domaine et très usitées. Enfin nous parlons des systèmes permettant de mettre en place des modèles pré-appris afin de mieux représenter les données contenues dans la parole.

Notre troisième et dernier chapitre de la première partie se concentre sur notre domaine d'étude, à savoir le Speech Emotion Recognition. Après avoir défini les contours de ce domaine, nous présentons les différents corpus utilisés dans ce domaine, tout en définissant les caractéristiques communes à ces corpus. Ensuite nous nous intéressons aux descripteurs, qui correspondent à la transformation du signal audio en un ensemble de données qui sont utilisables et compréhensibles par les systèmes de reconnaissance. Pour finir, nous parlons de la fusion des modalités acoustiques et linguistiques, et nous passons en revue une partie des campagnes AVEC, qui sert de baseline aux travaux de cette thèse.

Le quatrième chapitre, qui ouvre la seconde partie, se concentre sur la construction du corpus AlloSat, permettant l'étude de la satisfaction et de la frustration dans des conversations entre des clients et des conseillers de centres d'appels. Nous commençons par relater les techniques de recueil de données et notre mise en place de l'annotation. Une fois la création de ce corpus effectuée, nous avons fait des analyses afin de jauger sa pertinence et son utilisabilité. Enfin nous rapportons les modalités de sa diffusion.

Dans le cinquième chapitre, nous détaillons la mise en place des premiers systèmes de reconnaissances de l'émotion. Nous commençons par détailler les protocoles de reconnaissance que nous avons mis en place que ce soit pour les émotions discrètes ou les émotions continues. Nous détaillons les représentations acoustiques et les architectures neuronales mises en place puis nous discutons de la pertinence de la fonction de coût avant de proposer un post-traitement afin d'obtenir nos scores de référence pour le corpus AlloSat et pour le corpus SEWA.

Dans le sixième chapitre, nous entamons des discussions sur la pertinence de la fonction de coût, puis nous détaillons nos travaux sur la modalité linguistique. Nous détaillons ensuite les protocoles utilisés pour la fusion des modalités. Et enfin, nous présentons des travaux utilisant des descripteurs pré-entraînés, qui donnent les meilleurs résultats sur le corpus AlloSat.

Enfin, dans le septième et dernier chapitre, nous décrivons les travaux effectués sur la pertinence de systèmes de reconnaissance appris sur les annotations de chaque annotateur. Nous revenons également sur le succès de la modalité linguistique, que nous essayons d'expliquer en utilisant à la fois des procédés statistiques et des analyses linguistiques humaines. Enfin, nous présentons une ébauche de travaux qui met le genre au centre de la frustration.
