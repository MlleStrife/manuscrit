
\clearemptydoublepage
\chapter{Titre du premier chapitre}

\chapter{L'émotion}

L'étude de l'émotion humaine est à la croisée de plusieurs domaines dont notamment la psychologie, la physiologie et la linguistique. Sa définition et sa caractérisation est encore aujourd'hui source d'études.

\section{Définition de l'émotion}
La définition de l'émotion est exprimée différemment en fonction des domaines d'étude. Pour le grand public, le dictionnaire Le Robert (https://dictionnaire.lerobert.com/definition/emotion) définit trois sens du mot émotion :
\begin{itemize}
    \item État affectif intense, caractérisé par des troubles divers (pâleur, accélération du pouls, etc.). Par exemple : Être paralysé par l'émotion ; Tu nous as donné des émotions, tu nous as fait peur (familier).
    \item État affectif, plaisir ou douleur, nettement prononcé.
    \item Sensibilité. Par exemple : Interpréter une œuvre avec émotion.
\end{itemize}
Au sein de cette thèse, nous considérons l'émotion selon la deuxième définition : l'émotion est un état temporaire dans lequel se trouve une personne, causée par un sentiment vif ressenti habituellement en réponse à une stimulation de l'environnement. \\
Si on se réfère aux travaux de Darwin (1859, 1872), les émotions sont régies par sept principes.
\begin{itemize}
    \item les émotion sont innées : elles sont dues à l'Évolution et présentes dès la naissance. Elles se complexifient lorsque la personne grandit.
    \item elles suivent une continuité phylogénétique : les émotions sont aussi présentes chez les animaux proches de l'être humain, par exemple les primates.
    \item les émotions sont dénombrables : on peut caractériser chaque émotion par une des 8 catégories définies par Darwin (souffrances, abattement, joie, mauvaise humeur, haine, mépris, surprise et honte).
    \item elles sont analysables : on peut les caractériser en fonction de l'activité musculaire du visage.
    \item les émotions sont reconnaissables : les témoins reconnaissent l'émotion d'une personne en tant qu'information.
    \item elles sont universelles : comme elles viennent de l'Évolution, elles sont multi-culturelles et leur manifestation est reconnaissable par tous.
    \item elles sont actionnables : "le simple acte de simuler une expression tend à la faire naître dans notre esprit".
\end{itemize}
Les émotions servent donc à la survie de l’espèce et sont définies comme adaptatives. En effet, elles permettent d'adopter une réaction appropriée à un stimuli de l'environnement. Par exemple un danger se présente, concrètement un hippopotame. L'homme ressent une émotion en réponse à ce stimuli : la peur. Celle ci va activer toute une chaîne de réponses biologiques pour augmenter les chances de survie, notamment l'augmentation du rythme cardiaque pour mieux oxygéner les muscles et qui va permettre à l'homme de mieux attaquer ou fuir. \\
%Toutes les émotions auraient donc des réponses biologiques, nées de l'évolution, qui auraient pour but de favoriser la survie de l'individu.\\
%Elles sont également universelles donc multi-culturelles (1859 ; 1872), on considère que tous hommes, quelque soit son environnement et son origine, ressent des émotions.
Avec la naissance de la psychologie, de nombreuses théories ont émergées pour définir et caractériser l'émotion. Williams James (1884) définit lui l'émotion comme une conséquence de la réponse physiologique à un stimuli de l'environnement. Si une personne nous insulte, on ne crie pas parce qu'on est en colère, on est en colère parce qu'on crie. Cela implique que les émotions sont plutôt contrôlables, on peut les accentuer ou les inhiber.
Dans le cadre des travaux de Claude Bernard et Walter Cannon (1915, 1927), la notion d’homéostasie est définie: l'émotion est un processus qui permet au corps d'interrompre son fonctionnement normal pour concentrer ses ressources dans une réponse adaptée, principalement l'attaque ou la fuite avec notamment la libération d'adrénaline dans le corps. Ces travaux lui permettront également de situer la partie du cerveau responsable de l'émotion : les régions sous-corticales.
Les conclusions de ces travaux lanceront l'exploration du cerveau pour trouver les régions responsables des différentes émotions, amenant les neurosciences à s'emparer du domaine de l'émotion. \\
En parallèle de ces explorations, la notion d'activation va émerger avec Élisabeth Duffy (1934, 1941). En effet, maintenant que les émotions sont détectables dans le cerveau humain, elles vont être traduites par des mesures du potentiel d'activité. La théorie des émotions en catégories comme définie par Darwin est donc réévaluée. En effet, si toutes les émotions peuvent être mesurées en potentiel d'activité de certaines zones du cerveau, la frontière entre les émotions peut être plus perméable que préalablement définie.
Cette notion sera néanmoins plus difficile à démontrer que prévu, puisque toutes les mesures (électro-encéphalogramme, activité des viscères, tension musculaires) ne covarie pas et sont différentes d'un individu à un autre.
