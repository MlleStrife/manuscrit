\begin{table}[]
    \centering
    \begin{tabular}{| l | l | c | c | c |}
        \hline
        \textbf{Models} &\textbf{Features} &\multicolumn{3}{c|}{\textbf{SEWA}} \\ \cline{3-5}
        & &activation &valence &liking \\
        \hline
        \multicolumn{5}{|l|}{AVEC 2017~\cite{AVEC2017} : Sur les conversations allemandes} \\
        \hline
        SVR      &BoTW       &.373  &.390 &.314 \\
       \hline
       \multicolumn{5}{|l|}{Huang et al.~\cite{Huang2017} : Sur les conversations allemandes} \\
       \hline
       LSTM     &BoTW        &.451  &.518 &.473 \\
        \hline
       \multicolumn{5}{|l|}{Huang et al.~\cite{Huang2018} : Sur les conversations allemandes} \\
       \hline
       LSTM       &Word2Vec-300   &\textbf{.597}  &\textbf{.600} &.454 \\
       LSTM-2     &BoW            &  & &.407 \\
       LSTM-2     &Word2Vec       &  & &\textbf{.480} \\
       LSTM-2     &GloVe          &  & &.413 \\
       \hline
    \end{tabular}
    \caption{Compilation des scores de CCC sur l'ensemble de développement de SEWA sur les 3 dimensions : activation, valence et \textit{liking}. L'accronyme BoTW signifie \textit{Bag-of-text-words}, SVR signifie Support Vector Regression~\cite{Smola2004}, proche des SVM mais applicable à des problèmes de régression et donc à une annotation continue.}
    \label{tab:avectexte}
\end{table}
