\begin{table}[]
   \centering
   \begin{tabular}{| l | l | l |}
   \hline
       \textbf{Paramètres} &\textbf{fonctions appliquées} &\textbf{nb paramètres} \\
   \hline
       \multicolumn{3}{|l|}{\textbf{Paramètres fréquentiels : 20 ou 24 paramètres}} \\
   \hline
       Hauteur de la voix (Pitch) F0 &M, CV, P, RP, SLOPE &10 \\
       Micro-variations de F0 de la voix (Jitter) &M, CV &2 \\
       Frequence des Formants 1,2,3 &M, CV &6 \\
       \begin{tabular}[c]{@{}l@{}}Bande passante (Bandwidth) du \\
       premier Formant \end{tabular} &M, CV &2 \\
       Bande passante des Formants 2,3 \textbf{*} &M, CV &4 \\
   \hline
       \multicolumn{3}{|l|}{\textbf{Paramètres d'énergie et d'amplitude : 14 paramètres}} \\
   \hline
       Micro-variation d'énergie (Shimmer) &M, CV &2 \\
       Énergie perçue (Loudness) &M, CV, P, RP, SLOPE &10 \\
       \begin{tabular}[c]{@{}l@{}}Rapport Harmonique-Bruit \\
       (Harmonics-to-Noise Ratio) \end{tabular} &M, CV &2 \\
   \hline
       \multicolumn{3}{|l|}{\textbf{Paramètres spectraux : 22 ou 43 paramètres}} \\
   \hline
       Ratio Alpha &M, CV, MUN &3 \\
       Index de Hammarberg &M, CV, MUN &3 \\
       Pente spectrale : 0-500Hz &M, CV, MUN &3 \\
       Pente spectrale : 500-1500Hz &M, CV, MUN &3 \\
       \begin{tabular}[c]{@{}l@{}}Energie relative des \\
         Formants 1,2,3 \end{tabular} &M, CV &6 \\
       Différence harmonique H1-H2 &M, CV &2 \\
       Différence harmonique H1-A3 &M, CV &2 \\
       MFCC 1 à 4 \textbf{*} &M, CV, MCVV &16 \\
       Flux spectral \textbf{*} &M, CV, MUN, MCVV &5 \\
   \hline
       \multicolumn{3}{|l|}{\textbf{Paramètres temporels : 6 paramètres}} \\
   \hline
       \begin{tabular}[c]{@{}l@{}}Taux des pics d'énergie \\
       (Rate of loudness peaks) \end{tabular} & &1 \\
       Durée des segments parlés ($F0>0$) &M, V &2 \\
       Durée des segments non-parlés ($F0<0$) &M, V &2 \\
       \begin{tabular}[c]{@{}l@{}}Nombre de zones continues \\
       parlées par secondes \end{tabular} & &1  \\
   \hline
       %Équivalent du niveau du son \textcolor{red}{=> Intensité sonore ?} \textbf{*}& &1 \\
       Intensité sonore \textbf{*}& &1 \\
   \hline
   \end{tabular}
   \caption{Résumé des descripteurs de bas niveau (LLDs) utilisés dans les ensembles GeMAPS et eGeMAPS. \textbf{*} signale les descripteurs uniquement disponibles dans eGeMAPS. Légende des fonctions appliquées : \textbf{M} moyenne arithmétique, \textbf{CV} coefficient de variation, qui correspond à la variance normalisée par la moyenne, \textbf{V} variance, \textbf{P} percentiles : 20,50 et 80\%, \textbf{RP} l'intervalle des percentiles 20 à 80\%, \textbf{SLOPE} moyenne et variance des pente de montée/descente de signal, \textbf{MUN} moyenne arithmétique avec les parties sans parole, \textbf{MCVV} moyenne arithmétique et coefficient de variation des parties avec paroles.
   % \begin{itemize}
   %   \item \textbf{M} moyenne arithmétique,
   %   \item \textbf{CV} coefficient de variation, qui correspond à la variance normalisée par la moyenne,
   %   \item \textbf{V} variance,
   %   \item \textbf{P} percentiles : 20,50 et 80\%,
   %   \item \textbf{RP} l'intervalle des percentiles 20 à 80\%,
   %   \item \textbf{SLOPE} moyenne et variance des pente de montée/descente de signal,
   %   \item \textbf{MUN} moyenne arithmétique avec les parties sans parole.
   %   \item \textbf{MCVV} moyenne arithmétique et coefficient de variation des parties avec paroles.
   % \end{itemize}
   }
   \label{tab:egemaps}
\end{table}
