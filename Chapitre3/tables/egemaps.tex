\begin{table}[h]
   \centering
   \begin{tabular}{| l |}
   \hline
       Paramètres fréquentiels \\
   \hline
       Hauteur de la voix (Pitch)  \\
       Tremblement de la voix (Jitter) \\
       Frequence des Formants 1,2,3 \\
       Bande passante (Bandwidth) des Formants 1,2,3 \\
   \hline
       Paramètres d'énergie et d'amplitude \\
   \hline
       Scintillement (Shimmer) \\
       Volume (Loudness) \\
       Ratio Harmonique-Bruit (Harmonics-to-Noise Ratio) \\
   \hline
       Paramètres spectraux \\
   \hline
       Ratio Alpha \\
       Index de Hammarberg \\
       2 pentes spectrales : 0-500Hz et 500-1500Hz \\
       Energie relative des Formants 1,2,3 \\
       Différence harmonique H1-H2 et H1-A3 \\
       MFCC 1 à 4 \\
       Flux spectral \\
   \hline
       Paramètres temporels  \\
   \hline
       Taux des pics de volume (Rate of loudness peaks) \\
       Moyenne et variance des zones parlées \\
       Moyenne et variance des zones non parlées \\
       Nombre de zones continues parlées par secondes \\
   \hline

   \end{tabular}
   \caption{Résumé des descripteurs de bas niveau (LLDs) utilisés dans l'ensemble eGeMAPS.}
   \label{tab:egemaps}
\end{table}
