\begin{table}[]
    \centering
    \begin{tabular}{| l | l | c | c | c |}
        \hline
        \textbf{Models} &\textbf{Features} &\multicolumn{3}{c|}{\textbf{SEWA}} \\ \cline{3-5}
        & &activation &valence &liking \\
        \hline
        \multicolumn{5}{|l|}{AVEC 2017~\cite{AVEC2017} : Sur les conversations allemandes} \\
        \hline
        SVR      &BoAW~\cite{Schmitt2016} &.344  &.351 &.081 \\
       \hline
       \multicolumn{5}{|l|}{Huang et al.~\cite{Huang2017} : Sur les conversations allemandes} \\
       \hline
       LSTM     &eGeMAPS-88  &.506  &.455 &.193 \\
       LSTM     &IS10        &.465  &.440 &.227 \\
       LSTM     &Bottle-neck~\cite{Fer2015} &.533  &.466 &     \\
       LSTM     &Mfcc        &.341  &.421 &     \\
        \hline
        \multicolumn{5}{|l|}{AVEC 2018~\cite{AVEC2018} : Sur les conversations allemandes} \\
        \hline
        biLSTM-2 &eGeMAPS-88  &.124  &.112 &.001 \\
        biLSTM-2 &Mfcc        &.253  &.217 &.136 \\
         \hline
       \multicolumn{5}{|l|}{Huang et al.~\cite{Huang2018} : Sur les conversations allemandes} \\
       \hline
       LSTM     &eGeMAPS-88   &.497  &.438 &.281 \\
       LSTM     &eGeMAPS-89   &.520  &.461 &\textbf{.335} \\
       LSTM     &eGeMAPS-176  &.514  &.493 &.217 \\
        \hline
        \multicolumn{5}{|l|}{AVEC 2019~\cite{AVEC2019} : Sur les conversations allemandes et hongroises} \\
        \hline
        biLSTM-2 &eGeMAPS-88  &.371 &.286 &.159 \\
        biLSTM-2 &Mfcc        &.326 &.187 &.144 \\
         \hline
       \multicolumn{5}{|l|}{Schmitt et al.~\cite{Schmitt2019} : Sur les conversations allemandes} \\
       \hline
       CNN      &eGeMAPS-47    &\textbf{.571}  &.517 & \\
       biLSTM-4 &eGeMAPS-47    &.568  &\textbf{.561} & \\
       \hline
    \end{tabular}
    \caption{Compilation des scores de CCC sur l'ensemble de développement de SEWA sur les 3 dimensions : activation, valence et \textit{liking}. L'accronyme BoAW signifie \textit{Bag-of-audio-words}, SVR signifie Support Vector Regression~\cite{Smola2004}, proche des SVM mais applicable à des problèmes de régression et donc à une annotation continue. Les différents nombres associés aux features de type eGeMAPS dénotent de différentes configurations utilisées autour de ces sets : soit avec une sélection reduite des LLDs (47), soit avec l'ajout d'information sur le locuteur courant (89 et 176).}
    \label{tab:avec}
\end{table}
