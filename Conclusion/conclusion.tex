\chapter*{Conclusion}
\addcontentsline{toc}{chapter}{Conclusion}
\chaptermark{Conclusion}

Nous concluons ici sur les travaux réalisés pendant ces années de thèse et nous proposons quelques perspectives qui pourront peut être orienter de futurs travaux.

\subsubsection{Contexte}
Tout au long de cette thèse nous avons travaillé autour des émotions. Ce travail, inscrit dans un contexte industriel fort, permettra demain à l'entreprise Allo-Média de commercialiser une solution de reconnaissance automatique des émotions de satisfaction et de frustration dans des conversations issues de centre d'appels.

Nous avons tout d'abord défini, en accord avec le besoin industriel, les émotions pertinentes dans la relation clientèle. La satisfaction et la frustration sont des marqueurs forts, vecteurs de gain et de perte de clients par exemple : un client satisfait a plus de chance de continuer avec une entreprise alors qu'un client frustré peut aller voir la concurrence. Ces deux émotions, que nous traitons comme deux faces d'un même ensemble ne sont pas très présent dans la littérature si on considère nos obligations, à savoir la langue française et le canal téléphonique.

Une première partie de nos travaux a donc porté sur l'élaboration d'un corpus permettant de répondre à ce besoin industriel. En effet, il est difficile d'établir un quelconque système de reconnaissance sans avoir de données sur lesquelles s'appuyer. Nous avons donc construit le corpus AlloSat, qui est annoté en satisfaction et frustration de façon discrète et continue. Ce corpus de 37h d'audio, contient uniquement les voix des clients, puisque c'est sur ces derniers que nous focalisons notre attention. Il est disponible pour toute personne affiliée à un institut public de recherche sur simple demande aux personnes responsables de sa diffusion.

Une fois ce corpus dans nos mains, nous avons pu mettre en place des systèmes de reconnaissance de ces deux émotions. Nous avons tout d'abord travaillé sur les annotations discrètes de ce corpus. Ces études préliminaires nous ont permis de valider que des algorithmes d'apprentissage sont capable de retrouver les émotions de satisfaction et de frustration dans le langage.

Nous avons ensuite mis en place des systèmes de reconnaissance de l'émotion continue. Ce choix est justifié par les besoins industriels. En effet, nous avions besoin de connaître l'état émotionnel des clients mais aussi de pouvoir situer dans une conversation les instants de frustration. Cette reconnaissance s'appuie sur des réseaux de neurones dont nous avons comparés différentes architectures pour trouver la plus performante avec nos données. Nous nous sommes aussi beaucoup attardés sur la représentation des données et donc sur les features que nous donnons en entrée de la reconnaissance d'émotion.

Afin de juger de la pertinence mais surtout de la performance de nos systèmes, nous avons cherché à nous comparer à des expérimentations issues de l'état de l'art du domaine. Pour cela, nous avons étudié le corpus SEWA et nous avons mis en place une série d'expérimentations visant à reproduire nos travaux sur ce corpus. Bien que se ressemblant sur de nombreux points, les corpus restent très différents : la partie de SEWA utilisée est en allemand et hongrois, contient des conversations entre deux interlocuteurs, avec des durées de conversations plutôt homogènes, sur un sujet bien précis, n'utilisant pas le canal téléphonique, annotées en valence, activation et \textit{liking}... Alors qu'AlloSat est en français avec uniquement la voix d'un interlocuteur, dont les durées de conversations varient de quelques secondes à plusieurs dizaines de minutes, concernant plusieurs domaines d'activités, le tout avec un canal téléphonique, annotées en satisfaction/frustration. Autant de justifications possibles qui peuvent expliquer pourquoi ce qui fonctionne sur l'un de ces corpus, ne fonctionne pas forcément sur l'autre.

Pour aller plus loin, nous avons voulu exploré la modalité linguistique de notre corpus. En effet, jusque là, nous utilisions uniquement le signal audio. Grâce à un modèle de reconnaissance automatique de la parole, nous avons accès aux transcriptions de ce signal. Ainsi, il nous est possible de traiter indépendamment la modalité linguistique en se concentrant sur les mots. Cette étude a montré que, non content d'être pertinente, la modalité linguistique permet d'atteindre de meilleurs scores de reconnaissance. %Ce fait peut être expliqué par plusieurs phénomènes : tout d'abord, dans la nature intrinsèque des émotions que nous traitons, nous savons que la satisfaction et la frustration ont une valence fortement marquée (l'une positive, l'autre négative). La valence est très marquée dans la modalité linguistique.

Comme nous avons deux modalités que nous avons validé comme étant pertinentes, nous les avons fusionnés afin de questionner l'apport de chaque modalité et de savoir si elles sont redondantes ou non. Comme nous améliorons les scores de reconnaissance par ce procédé et que nous pouvons voir sur les courbes de prédictions, ces modalités semblent permettre de retrouver des informations complémentaires sur l'état émotionnel des individus.

De plus, nous nous sommes questionné sur des méthodes pour masquer un peu la faible quantité de données dont nous disposons. En effet, même si notre corpus est de taille respectable, il ne contient pas autant de données que des corpus traditionnellement utilisés pour alimenter des systèmes à base de réseaux de neurones. Notre solution a été d'utiliser des modèle pré-appris pour extraire des embeddings plus spécifiques. Cette technique, nouvelle au début de cette thèse, est aujourd'hui de plus en plus utilisée par de nombreux domaines. Nous avons pu montrer qu'elle est tout à fait pertinente avec notre corpus, nous permettant d’atteindre un score de $0.920$ su le test, ce qui correspond aujourd'hui au meilleur score atteint sur le corpus AlloSat.

Un autre aspect de notre travail a été de vouloir expliquer la performance de la modalité linguistique. En effet, nous ne pensions pas que cette modalité apporterait autant d'information pertinentes aux systèmes de reconnaissance. Nous avons donc voulu déceler les marqueurs de l'émotion dans les transcriptions. Pour cela, nous avons mis en place des méthodes statistiques et des écoutes humaines, effectuées par des informaticiens et un linguiste. Même si nous avons réussi à dégager quelques indicateurs, de plus amples investigations seraient à envisager.

Pour conclure, il me semblait important de vous parler de mon ressenti sur cette thèse. Bien que j'ai eu la chance de travailler sur un sujet passionnant, encadré par des personnes intéressées et très motivées par toutes ces problématiques, l'expérience de cette thèse me laisse un peu frustrée. J'aurai voulu approfondir certains aspects et surtout être en mesure de parler de mes travaux à d'autres personnes du domaine. Cependant la pandémie nous a tous fortement affectée. Couplée à quelques situations personnelles, je suis aujourd'hui à la fois triste de finir cette thèse sans avoir pu expérimenter plus et soulagée d'avoir réussi à la terminer. Je sais que la finalité d'une thèse est d'être sans fin, puisqu'elle soulève plus de questions qu'elle n'en solutionne, et à ce titre, j'espère pouvoir continuer à me poser des questions sur le sujet.

\subsubsection{Perspectives}
